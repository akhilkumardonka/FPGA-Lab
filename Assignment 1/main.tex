\documentclass{article}
\usepackage[utf8]{inputenc}
\usepackage[T1]{fontenc}
\usepackage{karnaugh-map}
\usetikzlibrary{calc}
\usepackage{placeins}

\usepackage{amsmath}
\usepackage{amsfonts}
\usepackage{multicol}
\usepackage{amssymb}
\usepackage[framed]{matlab-prettifier}
\usepackage{graphicx}
\usepackage[margin=0.75in]{geometry}
\usepackage{enumerate}
\usepackage{circuitikz}

\title{Assignment 1 | FPGA Lab}
\author{Akhil Kumar Donka}
\date{January 2022}

\begin{document}
\maketitle

\section{Question}

Write the Product of Sum form of function G(U,V,W) for the following truth table representation of F

    \begin{table} [h!]
    \centering
    \begin{tabular}{ | c | c | c | c | }
    \hline
    U & V & W & G \\ [0.5ex]
     \hline
    0 & 0 & 0 & 1 \\
    0 & 0 & 1 & 0 \\
    0 & 1 & 0 & 1 \\
    0 & 1 & 1 & 0 \\
    1 & 0 & 0 & 1 \\
    1 & 0 & 1 & 0 \\
    1 & 1 & 0 & 0 \\
    1 & 1 & 1 & 1 \\ [1ex]
    \hline
    \end{tabular}
    \end{table}

\section{Solution}

\subsection{KMAP Implementation}
Given truth table can be minimized using a KMap (Figure 1). Using implicants in figure, POS Terms obtained are : $(U + \overline{W}),(V + \overline{W}),(\overline{U} + \overline{V} + W)$

\begin{figure}[!htbp]
\centering
\resizebox{\columnwidth}{!}
{
\begin{karnaugh-map}[4][2][1][][]
    \minterms{0,2,4,7}
    \maxterms{1,3,5,6}
    \implicant{1}{3}
    \implicant{1}{5}
    \implicant{6}{6}
    \draw[color=black, ultra thin] (0,2) --
    node [pos=0.7, above right, anchor=south west] {$VW$} % YOU CAN CHANGE NAME OF VAR HERE, THE $X$ IS USED FOR ITALICS
    node [pos=0.7, below left, anchor=north east] {$U$} % SAME FOR THIS
    ++(135:1);
    \end{karnaugh-map}
}
\caption{POS for G}
\label{fig:kmap_G_pos}
\end{figure}
\FloatBarrier

\subsection{Minimized POS Expression}
G = $(U + \overline{W}).(V + \overline{W}).(\overline{U} + \overline{V} + W)$

\subsection{NAND Logic Implementation}

To implement it using NAND Logic, we first convert it into SOP form, which gives :

\begin{center}
   $\overline{V}.\overline{W} + \overline{U}.\overline{W} + U.V.W$\\
   \vspace{5pt}
   $(\overline{V} + \overline{U}).\overline{W} + U.V.W$\\
   \vspace{5pt}
   $\overline{V.U}.\overline{W} + U.V.W$\\
\end{center}
The last expression can be implemented using only NAND Gates.

\begin{center}
\begin{circuitikz}
\draw
(0,0)node[nand port](mynand1){}

(0,2)node[nand port](mynand3){}

(2,2)node[nand port](mynand4){}

(2,0)node[nand port](mynand5){}

(4,0)node[nand port](mynand6){}

(6,1)node[nand port](mynand7){}

(mynand3.out)|-(mynand4.in 1)
(mynand1.out)|-(mynand4.in 2)
(mynand4.out)|-(mynand7.in 1)
(mynand6.out)|-(mynand7.in 2)
(mynand5.out)|-(mynand6.in 1)
(mynand1.out)|-(mynand5.in 1)
(mynand1.out)|-(mynand5.in 2)
(mynand3.in 1)|-(mynand3.in 2)

(mynand1.in 1) node (A1)     [anchor=east]           {U}
(mynand1.in 2) node (A2)     [anchor=east]           {V}
(mynand6.in 2) node (A3)     [anchor=east]           {W}
(mynand7.out) node (A4)     [anchor=east, xshift=0.5cm]           {G}
(mynand3.in 2) node (B1)     [anchor=east]  {W};

\end{circuitikz}
\end{center}

\end{document}
